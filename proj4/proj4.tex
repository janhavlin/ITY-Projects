% 4. projekt do předmětu ITY
% Autor: Jan Havlín, 1BIT, xhavli47@stud.fit.vutbr.cz
% Datum: 16. 4. 2018
% Popis: Bibliografické citace

\documentclass[11pt,a4paper]{article}
\usepackage{times}
\usepackage[left=2cm,text={17cm,24cm},top=3cm]{geometry}
\usepackage[utf8]{inputenc}
\usepackage[czech]{babel}
\usepackage[T1]{fontenc}

\bibliographystyle{czplain}

\begin{document}
\begin{titlepage}
\begin{center}
	\Huge
	\textsc{Vysoké učení technické v~Brně}\\
	\huge
	\textsc{Fakulta informačních technologií} \\
	\vspace{\stretch{0.382}}
	\LARGE
	Typografie a~publikování\,--\,3. projekt \\
	\Huge
	Bibliografické citace
	\vspace{\stretch{0.618}}
\end{center}
{\Large \today \hfill Jan Havlín}
\end{titlepage}
\section{Úvod}
Typografie se zabývá vzhledem tiskového písma. V~dnešní době se s~touto formou vyjadřování setkáváme velmi často, proto si typografie zaslouží nějakou pozornost \cite{leaning:begin}. Dobrá typografie vzniká věnováním pozornosti detailům. Typografické aspekty písma ovlivňují čtenářovu schopnost efektivního čtení obsahu. Jeden z~nejdůležitějších aspektů je výběr fontu \cite{journal:graphic}. Čitelnost potisků produktech má významný vliv na jejich prodej, lidé si méně pravděpodobněji koupí produkt, na kterém je nečitelný font \cite{journal:advertising}.

\section{Historie typografie}
Písmo a~typografie se vyvinulo z~piktogramů -- zjednodušené kresby, jež reprezentovaly konkrétní myšlenky. Inkové a~Egypťané používali systém nazývaný hieroglyfy \cite{master:historie}. Moderní období typografie vzniká s~vynalezením knihtisku. Hlavní vývoj technik tisku a~písem ovšem proběhl v~19. a~20. století \cite{bachelor:historie}.

\section{Pravidla typografie}
Několik typografických pravidel:
\begin{itemize}
\item{Pro sazbu spojovníku se používá znak \uv{\texttt{-}}. Pokud se spojovník nachází na konci řádku, musíme ho zopakovat na začátku dalšího řádku \cite{cstug:spojovnik}.}

\item{Nezlomitelná mezera zabraňuje zalomení řádku. Používáme ji v~zápisu mezi číslem a~jednotkou (např. 10\,kg), v~psaní data (např. 15.\,4.\,2018) nebo při oddělování tisíců (např. 1\,000\,000) \cite{kerslager:pravidla}.}

\item{Symboly jako čárka, tečka, vykřičník, otazník, středník, dvojtečka se píší těsně za konec slova. Za nimi následuje mezera \cite{wiki:pravidla}.}
\end{itemize}

\section{Systém \LaTeX}
\LaTeX je nástroj pro přípravu textů hojně využívaný vědci a~akademiky. Je užitečný zejména pro sázení technických materiálů \cite{helmut:guidetolatex}. Naučit se pracovat s~\LaTeX em není tak snadné. Pro začátečníky i~pokročilé ovšem existuje řada knih, které se zabývají nejrůznějšími problémy tohoto sázecího prostředí \cite{gratzer:practical}.

\newpage
\bibliography{citace}
\end{document}