\documentclass[11pt,twocolumn,a4paper]{article}
\usepackage{times}
\usepackage[left=1.5cm,text={18cm,25cm},top=2.5cm]{geometry}
\usepackage[utf8]{inputenc}
\usepackage[czech]{babel}
\usepackage[IL2]{fontenc}
\usepackage{amssymb}
\usepackage{amsmath}
\usepackage{amsthm}
\usepackage{amsfonts}

\theoremstyle{definition}
\newtheorem{definition}{Definice}
\theoremstyle{plain}
\newtheorem{sentence}{Věta}

\begin{document}
\begin{titlepage}
\begin{center}
	\Huge
	\textsc{Fakulta informačních technologií} \\
	\textsc{Vysoké učení technické v~Brně}\\
	\vspace{\stretch{0.382}}
	\LARGE
	Typografie a publikování\,--\,2. projekt \\
	Sazba dokumentů a matematických výrazů
	\vspace{\stretch{0.618}}
\end{center}
{\Large 2018 \hfill Jan Havlín (xhavli47)}
\end{titlepage}

\section*{Úvod}
V~této úloze si vyzkoušíme sazbu titulní strany, matematických vzorců, prostředí a dalších textových struktur obvyklých pro technicky zaměřené texty (například rovnice~(\ref{eq:1}) nebo Definice \ref{def:turinguv} na straně \pageref{def:turinguv}). Rovněž si vyzkoušíme používání odkazů \verb|\ref| a \verb|\pageref|.\par Na titulní straně je využito sázení nadpisu podle optického středu s~využitím zlatého řezu. Tento postup byl probírán na přednášce. Dále je použito odřádkování se zadanou relativní velikostí 0.4em a 0.3em.

\section{Matematický text}
Nejprve se podíváme na sázení matematických symbolů a~výrazů v~plynulém textu včetně sazby definic a vět s~využitím balíku \texttt{amsthm}. Rovněž použijeme poznámku pod čarou s~použitím příkazu \verb|\footnote|. Někdy je vhodné použít konstrukci \verb|${}$|, která říká, že matematický text nemá být zalomen.\par 

\noindent
\begin{definition}\label{def:turinguv}
Turingův stroj \emph{(TS) je definován jako šestice tvaru $M = (Q, \Sigma, \Gamma, \delta, q_0, q_F )$, kde:}

\begin{itemize}
\item $Q$ \emph{je konečná množina} vnitřních (řídicích) stavů,
\item $\Sigma$ \emph{je konečná množina symbolů nazývaná} vstupní
abeceda, $\Delta \notin \Sigma$,
\item $\Gamma$ \emph{je konečná množina symbolů}, $\Sigma \subset \Gamma$, $\Delta \in \Gamma$,
\emph{nazývaná} pásková abeceda,
\item $\delta : (Q\backslash \{q_F \})\times \Gamma \rightarrow Q \times (\Gamma \cup \{L, R\}$), \emph{kde} $L, R \notin \Gamma$, \emph{je parciální} přechodová funkce,
\item $q_0$ \emph{je} počáteční stav, $q_0 \in Q$ \emph a
\item $q_F$ \emph{je} koncový stav, $q_F \in Q$.
\end{itemize}

\end{definition}
Symbol $\Delta$ značí tzv. \emph{blank} (prázdný symbol), který se vyskytuje na místech pásky, která nebyla ještě použita (může ale být na pásku zapsán i později). \par

\emph{Konfigurace pásky} se skládá z~nekonečného řetězce, který reprezentuje obsah pásky a pozice hlavy na tomto řetězci. Jedná se o~prvek množiny  $\{\gamma \Delta^\omega\,|\,\gamma \in \Gamma^*\}\times \mathbb{N}$.\footnote{Pro libovolnou abecedu $\Sigma$ je $\Sigma^\omega$ množina všech  \emph{nekonečných} řetězců nad $\Sigma$, tj.\,nekonečných posloupností symbolů ze $\Sigma$. Pro připomenutí: $\Sigma^*$ je množina všech \emph{konečných} řetězců nad $\Sigma$.}

\noindent \emph{Konfiguraci pásky} obvykle zapisujeme jako $\Delta xyz\underline{z}x \Delta$... (podtržení značí pozici hlavy). \emph{Konfigurace stroje} je pak dána stavem řízení a konfigurací pásky. Formálně se jedná o~prvek množiny $Q \times \{\gamma \Delta^\omega\,|\,\gamma \in \Gamma^*\}\times \mathbb{N}$.


\subsection{Podsekce obsahující větu a odkaz}
\begin{definition}\label{def:retezec}
Řetězec $w$ nad abecedou $\Sigma$ je přijat TS $M$ \emph{jestliže $M$ při aktivaci z~počáteční konfigurace pásky} $\underline{\Delta}w\Delta$... \emph{a počátečního stavu} $q_0$ \emph{zastaví přechodem do koncového stavu} $q_F$\emph{, tj.} $(q_0, \Delta w\Delta^\omega, 0) \overset{*}{\underset{M}{\vdash}} (q_F , \gamma, n)$ \emph{pro nějaké} $\gamma\in\Gamma^*$ \emph{a} $n \in \mathbb{N}$.

\emph{Množinu} $L(M) = \{w\,|\,w$ \emph{je přijat TS} $M \} \subseteq\Sigma^*$ \emph{nazýváme} jazyk přijímaný TS $M$.

\end{definition}

Nyní si vyzkoušíme sazbu vět a důkazů opět s~použitím balíku \texttt{amsthm}.

\begin{sentence} Třída jazyků, které jsou přijímány TS, odpovídá \emph{rekurzivně vyčíslitelným jazykům.}\end{sentence}

\begin{proof} V~důkaze vyjdeme z~Definice \ref{def:turinguv} a \ref{def:retezec}. \end{proof}

\section{Rovnice a odkazy}
Složitější matematické formulace sázíme mimo plynulý text. Lze umístit několik výrazů na jeden řádek, ale pak je třeba tyto vhodně oddělit, například příkazem \verb|\quad|.\par

$$\sqrt[i]{x_i^3}\quad \text{kde } x_i \text{ je }i\text{-té sudé číslo}\quad y_i^{2\cdot y_i} \neq y_i^{y_i^{y_i}}$$

V~rovnici (1) jsou využity tři typy závorek s~různou explicitně definovanou velikostí. \par

\begin{align} \label{eq:1}
x \quad &= \quad \left\{ \Big( \big[ a+b \big] *c \Big)^d \oplus 1 \right\}\\
y \quad &= \quad \lim_{x \to\infty} \frac{\sin^2 x + \cos^2 x}{\frac{1}{\log_{10}x}} \nonumber
\end{align}

V~této větě vidíme, jak vypadá implicitní vysázení limity $\lim_{n \to\infty} f(n)$ v~normálním odstavci textu. Podobně je to i s~dalšími symboly jako $\sum_{i=1}^n 2^i$ či $\bigcup_{A \in \mathcal{B}}A$. V~případě vzorců $\lim\limits_{n \to\infty} f(n)$ a $\sum\limits_{i=1}^n 2^i$ jsme si vynutili méně úspornou sazbu příkazem \verb|\limits|.\par

\begin{align}
\int\limits_a^b f(x)\,\mathrm{d}x \quad &= \quad -\int_b^a g(x)\,\mathrm{d}x \\
\overline{\overline{A \lor B}} \quad &\Leftrightarrow \quad \overline{\overline{A} \land \overline{B}}
\end{align}

\section{Matice}
Pro sázení matic se velmi často používá prostředí \verb|array| a závorky (\verb|\left|, \verb|\right|).

$$
\left(
\begin{array}{ccc}
a+b & \widehat{\xi + \omega} & \hat{\pi} \\
\vec{a} & \overset{\longleftrightarrow}{AC} & \beta
\end{array}
\right)
= 1 \iff \mathbb{Q} = \mathbb{R}
$$

$$
\text{A} = 
\left\rVert
\begin{array}{cccc}
a_{11} & a_{12} & \ldots & a_{1n} \\
a_{21} & a_{22} & \ldots & a_{2n} \\
\vdots & \vdots & \ddots & \vdots \\
a_{m1} & a_{m2} & \ldots & a_{mn} \\
\end{array}
\right\rVert
=
\left|
\begin{array}{cc}
t & u \\
v & w
\end{array}
\right|
= tw-uv
$$

Prostředí \verb|array| lze úspěšně využít i jinde.

$$
\binom{n}{k}
=
\left\{
\begin{array}{ll}
\frac{n!}{k!(n-k)!} & \text{pro } 0 \leq k \leq n \\
0 & \text{pro } k < 0 \text{ nebo } k > n
\end{array}
\right.
$$

\section{Závěrem}
V~případě, že budete potřebovat vyjádřit matematickou konstrukci nebo symbol a nebude se Vám dařit jej nalézt v~samotném \LaTeX u, doporučuji prostudovat možnosti balíku maker \AmS-\LaTeX.
\end{document}